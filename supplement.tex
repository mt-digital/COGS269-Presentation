% Supplement to Two models of the evolution of (non-)Universal Grammar
% COGS 269 April 5, 2017
% Matthew Turner

\documentclass[11pt,letterpaper]{article}

\usepackage{pslatex}
\usepackage{apacite}
\usepackage{amsmath}
\usepackage{graphicx}
\usepackage[margin=1.35in]{geometry}


\title{A semantic network approach to analysis of cable news}
\author{Matthew Turner}
\date{March 31, 2017}

\begin{document}

\maketitle


\section{Introduction}
\label{sec:Introduction}

For over two millenia the best thinkers have wondered about language. Is 
language mainly biological phenomenon that emerged when a ``language organ'' 
evolved to a 
critical point? Or, is language a tool, like a bow and arrow, or a computer? 
The two models and their associated papers reviewed here each
aim to support these two opposing views. As shown in Section \ref{sec:PropagatorEquivalence},
the two models are, in a general sense, mathematically equivalent. Both
consider stochastic transitions from one state of language use in a human
population to another. However, the mechanism that gives rise to the equations 
are different. 

\subsection{Motivation}
\label{sub:Motivation}

The foundations of language are important in and of themselves. Clearly in order
to understand the mind, as a whole, and the behavior of minds in society,
we must understand the biological and cultural foundations of language. However,
I believe that the papers presented here point towards a general method for
studying language development over time. While these papers study the evolution 
of grammar, which is roughly the structure of utterances, they both 
demonstrate a general method for agent- and population-based modeling of 
language change over any timescale. 

My research focuses on changes in pragmatics of language in relationship
to culture. For example, I am studying the effect of the beginning and ending
of the series of presidential debates on the use of metaphorical violence on
cable news. There are many cultural and psychological issues going on, but
if combined in a smart way, these could be included in an agent-based model
that mimicks the real-world system. In another project, I am investigating
the differences and dynamics semantic networks as revealed by 
vector space models of language use by 24-hour news stations. We can see in
near real time semantic relationships changing, and we can quantify the difference
in meaning of flashpoint words like ``Islam'', ``conservative'', ``liberal'',
or ``democracy''. 

\subsection{Outline}
\label{sub:Outline}





\section{Equivalence of the two propagators}
\label{sec:PropagatorEquivalence}

The \citeA{Nowak2001} equation of population dynamics at any given timestep
is

\begin{equation}
  \dot{x}_i = \sum_{j} x_j f_j Q_{ji} - \phi x_i
  \label{eq:NowakDynamics}
\end{equation}

\noindent
\citeA{Thompson2016} use the dynamics equation

\begin{equation}
  g^{(t+1)}_i = \frac{1}{\phi} \sum_{j} g_j^t f_j m_{ij}
  \label{eq:ThompsonDynamics}
\end{equation}

\noindent
I will show these can both be written in a general form 

\begin{equation}
  g_i^{t+1} = \sum_j A_{ji} g_i^t
\end{equation}

This demonstration illustrates the commonality between the two models. They
both assume there is some probabilistic transition 
from one ``grammar'' or ``language type'' to another. The matrix $A_{ji}$ is
then the transition matrix. \citeA{Nowak2001} consider under what conditions
does their dynamics equation, Equation \ref{eq:NowakDynamics}, have stable
solutions. For these stable solutions, they then determine what the solutions
have to say to either support or disconfirm the existence of a Universal Grammar 
that has evolved.

On the other hand, \citeA{Thompson2016} consider two models with the same
dynamical equation---the difference between the models mainly concerns the
nature of $m_{ji}$ in their dynamics equation, Equation 
\ref{eq:ThompsonDynamics}. Within each of the two models, Thompson and coworkers
evaluate what steady states emerge as $t \rightarrow \infty$ based on
Bayesian probabilities and infinite population ($n \rightarrow \infty$). 
They also consider finite $n$ and implement agent-based models with Bayesian
learners.

\subsection{Derivation}
\label{sub:Derivation}

The derivation for Equation \ref{eq:ThompsonDynamics} is obvious, we just have $A_{ji} = \frac{1}{\phi} f_j m_{ij}$.
To obtain this general form from Equation \ref{eq:NowakDynamics} we use the
Kronecker delta function defined as 

\[
  \delta_{ij} = \begin{cases}
    1~\mathrm{if}~i = j \\
    0~\mathrm{otherwise} 
  \end{cases}
\]

This allows us to write 

\[
  x_i = \sum_j \delta_{ji} x_j
\]

Then we have 

\begin{align}
  \begin{split}
  \dot{x}_i &= \sum_j x_j f_j Q_{ji} - \phi \sum_j \delta_{ji} x_j \\
            &= \sum_j x_j \left( f_j Q_{ji} - \phi \delta_{ji} \right)
  \end{split}
\end{align}

Now since $\dot{x}_i$ is the time derivative of the population frequency 
speaking with grammar $G_i$, the frequency at time $t+1$ can be written

\begin{align}
  \begin{split}
  x^{t+1}_i &= x^{t}_i + \dot{x}_i \\
            &= \sum_j \delta_{ji} x^{t}_j + \sum_j x^t_j \left( f_j Q_{ji} - \phi \delta_{ji} \right) \\
            &= \sum_j x^t_j \left( f_j Q_{ji} + (1 - \phi) \delta_{ji} \right) \\
            &= \sum_j A_{ji} x^t_j
  \end{split}
\end{align}

\noindent so for the Nowak dynamics, 
$A_{ji} = \left( f_j Q_{ji} + (1 - \phi) \delta_{ji} \right)$
\citeA{Nowak2001} uses $x^t_i$ to represent the frequency of the population
speaking with grammar $G_i$ while $g^t_i$ is ``the proportion of the population 
at generation $t$ made up by learners with prior $\alpha_i$.'' This will be
explained in more detail later.


\bibliographystyle{apacite}

\setlength{\bibleftmargin}{.125in}
\setlength{\bibindent}{-\bibleftmargin}

\bibliography{bib}

\end{document}
